\documentclass[12pt]{article}
\usepackage{amsthm}

\newtheorem{defn}{Definici\'on}

\begin{document}

\title{Tarea Programada N\'umero 1}
\author{Jos\'e Castro, Inteligencia Artificial}
\date{}
\maketitle

\section{Introducci\'on}
El objetivo de esta tarea es que el estudiante aprenda el manejo de herramientas para desarrollo en c\'odigo abierto, se familiarice con el lenguaje Python, y aprenda los algoritmos principales para inferencia en redes bayesianas mediante su colaboraci�n en la implementaci�n de una biblioteca de redes bayesianas en Python.

\section{Objetivos espec�ficos}
\begin{itemize}
	\item Obtener cuentas de github.com, gravatar.com, y aws.amazon.com
	\item Montar y configurar un nodo en la nube de Amazon Web Services
	\item Generar un fork local en su nodo Amazon del repositorio de python para la biblioteca que se implementa en la tarea
	\item Colaborar y aportar al c\'odigo de la biblioteca
\end{itemize}

\section{Log\'istica}
La tarea se efectuar\'a de manera colaborativo sobre un repositorio en \texttt{github.com}, en el enunciado de la tarea (este documento) se establece una serie de tareas que se deben llevar a cabo, cada estudiante tomar\'a una tarea, la trabajar\'a y la someter\'a para ser aceptada en el c\'odigo del sistema.

Esta especificaci\'on, escrita en \LaTeX{ } tambi\'en se mantendr\'a al dia en el repositorio de \texttt{github.com}, as\'i que es responsabilidad del estudiante estar al dia con el estado de la especificaci\'on.

\section{Actividades}
\begin{defn}
Una variable discreta  es un objeto \texttt{python} que se podr\'a leer de un archivo JSON y al cual se le puede asociar un conjunto de valores, cada valor puede opcionalmente contar con una descripci�n. Por ejemplo:

\begin{verbatim}
	trabaja = { 
		"nombre" : "Trabaja",
		"descripcion" : "indica si la persona trabaja"
		"cardinalidad" : 2,
		"dominio" : ["Falso", "Verdadero"]
		}
	Historial = {
		"nombre" : "Historial de Credito",
		"descripcion" : "indicador de confianza en cliente",
		"cardinalidad" : 5,
		"dominio" : ["Excelente", "Bueno", "Regular", "Malo", "Pesimo"]
		}
\end{verbatim}
\end{defn}

\subsection{Actividad 1}
Escribir la clase en \texttt{python} para la variable discreta. Debe tener m\'etodos para leer y guardar
a un stream su configuraci\'on en JSON.

\subsection{Actividad 2}
Implementar el esqueleto de las funciones y clases que se desarrollan en la tarea.
 
\subsection{Actividad 3}
Implementar el Factor en \texttt{python}, el factor es clase que implementa una funci\'on de un conjunto de variables discretas,
y que mapea los valores de las variables  a valores num\'ericos (no necesariamente una probabilidad).

OBSERVACIONES
\begin{itemize}
	\item El orden de las variables en el factor es relevante, y debe ser tomado en consideraci\'on.
	\item El factor debe guardar las variables en un vector.
	\item El factor debe guardar expl\'icitamente la cardinalidad de las variables en un vector.
	\item El factor debe implementar los valores como un vector (eventualmente un numpy array), y dentro del factor
		se ha de incluir un diccionario que mapa los nombres de los factores al indice del vector.
	\item El constructor del Factor debe asignarle 0 por defecto a los valores del vector.
	\item El factor se debe poder inicializar con una lista de valores para el vector.
\end{itemize}

\subsection{Actividad 4}
Implementar un test suite para la tarea (casos de prueba) que se pueda correr cada vez que se le hace una modificaci\'on para
verificar que no se ha quebrado la funcionalidad.

\subsection{Actividad 5}
Implementar una funci\'on utilitaria \texttt{assignmentToIndex(assign, dimension) = idx} que convierte los indices de una variable
a un indice lineal.
\begin{verbatim}
>>> assignmentToIndex([0,1,2],[3,1,3])
5
\end{verbatim}
La l\'ogica del map se puede encontrar si se expanden los \'indices en una tabla:
\begin{center}
\begin{verbatim}
0 0 0 = 0
0 0 1 = 1
0 0 2 = 2
0 1 0 = 3
0 1 1 = 4
0 1 2 = 5
\end{verbatim}
\end{center}

\subsection{Actividad 6}
Implementar la funci\'on \texttt{indexToAssignment}, funci\'on inversa de \texttt{assignmentToIndex}
\begin{verbatim}
>>> indexToAssignment(5,[3,1,3])
[0,1,2]
\end{verbatim}

\end{document}
